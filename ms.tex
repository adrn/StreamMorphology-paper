%\documentclass{emulateapj}
\documentclass[letterpaper,12pt,preprint]{aastex}

% packages
\usepackage{amssymb,amsmath,amsbsy}

% commands
\newcommand{\given}{\,|\,}
\newcommand{\dd}{\mathrm{d}}
\newcommand{\transpose}[1]{{#1}^{\mathsf{T}}}
\newcommand{\inverse}[1]{{#1}^{-1}}
\newcommand{\msun}{\mathrm{M}_\odot}

\begin{document}

\title{Tidal streams in triaxial systems}
\author{Adrian M. Price-Whelan\altaffilmark{\colum,\adrn}}

% Affiliations
\newcommand{\colum}{1}
\newcommand{\adrn}{2}
\altaffiltext{\colum}{Department of Astronomy, 
		              Columbia University, 
		              550 W 120th St., 
		              New York, NY 10027, USA}
\altaffiltext{\adrn}{To whom correspondence should be addressed: adrn@astro.columbia.edu}

\begin{abstract}
% Context
% Aims
% Methods
% Results
% Conclusions
\end{abstract}

\keywords{
}

%\begin{figure*}[!h]
%\begin{center}
%\includegraphics[width=\textwidth]{PATH}
%\caption{ CAPTION }\label{fig:FIGNAME}
%\end{center}
%\end{figure*}

\section{Introduction}\label{sec:introduction}

The halos of galaxies are filled with substructure...

As a satellite galaxy or globular cluster orbits within a parent galaxy, its mass is eroded due to the tidal forces of the host galaxy. For (typically) many orbits, these tidally stripped stars remain spatially coherent and

Of particular interest are such cold, young dynamical structures --- ``tidal streams'' -- which provide useful information about the mass distributions at large distances in the galaxies within which they orbit. [At these distances, disk / classic baryonic components can't help you, mostly dark matter]

The formation and morphology of tidal streams has been studied extensively in a variety of assumed potential forms. [Stream formation 

\section{Methods}\label{sec:methods}
\subsection{Potential form}

\section{Experiments}\label{sec:experiments}

\section{Discussion}\label{sec:discussion}

\acknowledgements
APW is supported by a National Science Foundation Graduate Research Fellowship under Grant No.\ 11-44155. 
This research made use of Astropy, a community-developed core \texttt{Python} package for Astronomy \citep{astropy13}.
This work additionally relied on Columbia University's \emph{Hotfoot} and \emph{Yeti} compute clusters, and we acknowledge the Columbia HPC support staff for assistance. \\

\bibliographystyle{apj}
\bibliography{refs}

\end{document}
