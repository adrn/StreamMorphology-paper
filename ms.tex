%\documentclass{emulateapj}
\documentclass[letterpaper,12pt,preprint]{aastex}

% packages
\usepackage{amssymb,amsmath,amsbsy}
\usepackage{booktabs}
\usepackage{bbold}
\usepackage{mathrsfs}
\usepackage{graphicx} 
\usepackage[backref,breaklinks,colorlinks,citecolor=blue]{hyperref}
\usepackage[yyyymmdd]{datetime}
\renewcommand{\dateseparator}{-}

% commands
\newcommand{\given}{\,|\,}
\newcommand{\dd}{\mathrm{d}}
\newcommand{\transpose}[1]{{#1}^{\mathsf{T}}}
\newcommand{\inverse}[1]{{#1}^{-1}}
\newcommand{\mean}[1]{\left< #1 \right>}
\newcommand{\msun}{\ensuremath{\mathrm{M}_\odot}}
\newcommand{\bs}[1]{\boldsymbol{#1}}
\newcommand{\ident}{\mathbb{1}}
\newcommand{\inttime}{t_{\rm int}}
\newcommand{\periods}{\ensuremath{{\rm T}}}

\newcommand{\project}[1]{\textsl{#1}}
\newcommand{\superfreq}{\project{SuperFreq}}

% TO DO
\newcommand{\todo}[2]{{\color{red} TODO: (\MakeUppercase{#1}) #2}}

\newcommand{\act}{J}
%\newcommand{\jac}{\bs{\rm J}}
%\newcommand{\jac}{\bs{G}}
\newcommand{\jac}{\mathscr{J}}

\begin{document}

\title{Chaotic mixing of tidal debris}
\author{Adrian M. Price-Whelan\altaffilmark{\colum,\adrn},
	    Kathryn V. Johnston\altaffilmark{\colum},
	    Monica Valluri\altaffilmark{\mich},
	    Sarah Pearson\altaffilmark{\colum},
	    Andreas H. W. K\"upper\altaffilmark{\colum},
	    David W. Hogg\altaffilmark{\nyu,\cds,\mpia}}
\date{\centering \today}

% Affiliations
\newcommand{\colum}{1}
\newcommand{\adrn}{2}
\newcommand{\mich}{3}
\newcommand{\nyu}{4}
\newcommand{\cds}{5}
\newcommand{\mpia}{6}
\altaffiltext{\colum}{Department of Astronomy, 
		              Columbia University, 
		              550 W 120th St., 
		              New York, NY 10027, USA}
\altaffiltext{\adrn}{To whom correspondence should be addressed: adrn@astro.columbia.edu}
\altaffiltext{\mich}{Department of Astronomy, 
			   University of Michigan,
			   Ann Arbor, MI 48109, USA}
\altaffiltext{\nyu}{Center for Cosmology and Particle Physics,
                      Department of Physics, New York University,
                      4 Washington Place, New York, NY, 10003, USA}
\altaffiltext{\cds}{Center for Data Science,
                      New York University,
                      4 Washington Place, New York, NY, 10003, USA}
\altaffiltext{\mpia}{Max-Planck-Institut f\"ur Astronomie,
                     K\"onigstuhl 17, D-69117 Heidelberg, Germany}

\begin{abstract}

\todo{apw}{write the abstract...}

% Context
%The large-scale, smooth components of galactic gravitational potentials are thought to be triaxial in shape. 
%Triaxial potentials that match observational constraints and measurements of the radial profile and axis ratios of dark matter halos in cosmological simulations are almost certainly not globally integrable and therefore contain an appreciable number of chaotic orbits.
%However, estimates of the timescale over which chaos is likely to be important (e.g., the Lyapunov time) suggest that chaos may not be dynamically relevant over physically meaningful times.
% Aims:

% Methods: 

%% Results:
% We find that indicators such as the Lyapunov time or the frequency diffusion time of an orbit are not good predictors for the short-time density evolution of ensembles of orbits (e.g., tidal debris) around this parent orbit.
% We propose a new measure for evaluating whether chaotic diffusion is important for tidal debris with a given spread in orbital properties over a given period of time: the variance of the fundamental frequencies of the ensemble orbits over short timescales is a stronger indicator of the density evolution.
% Conclusions:

\end{abstract}

\keywords{
}

\section{Introduction}\label{sec:introduction}

The dark matter halos of galaxies are thought to be triaxial in shape. However, despite suggestive evidence from a range of complimentary observational methods, this fundamental prediction from $\Lambda$CDM cosmology has not been conclusively verified. Around other galaxies, it is generally hard to measure the 3D mass profile because they are seen in projection. From the Earth's position within the Milky Way, our view of our own halo and proximity gives us a unique chance to directly measure the 6D positions of stars and model the shape of the mass distribution at large radii. The Milky Way halo has a low density of visible tracers, but luckily many of the halo stars are likely associated with debris stripped from stellar systems and thus contain extra information.

As a satellite galaxy or globular cluster orbits within some larger system, mass is eroded due to the tidal forces of the host galaxy potential. Along regular, mildly-eccentric orbits, mass is disrupted with small spreads in orbital properties (e.g., energy, angular momentum). Once the debris has evolved far enough from the progenitor system that the self-gravity of the progenitor can be ignored (usually a fast process relative to the orbital time), the stars evolve essentially as an ensemble of test particle orbits in the potential of the host system. The debris remains coherent as a tidal stream if the phase-mixing time-scale is long: a small ensemble of regular orbits reaches a fully phase-mixed state in a timescale $\approx\sigma_\Omega^{-1}$, where $\sigma_\Omega$ is the dispersion in fundamental frequencies of the ensemble (tidal debris from a globular cluster typically has frequency spreads $\approx$0.1--1\%, so it can take hundreds to thousands of orbital periods to fully phase-mix). The ensemble spreads due to sheering from slight variations in their fundamental frequencies which, for tidal streams, preferentially occurs along one dimension \citep{merritt96, helmi99}. 

The morphological (density) evolution of the tidal debris therefore depends only on the spread of orbital properties (e.g., actions or frequencies) of the debris and the orbit of the progenitor system, both of which are also determined by the shape and radial profile of the gravitational potential of the host galaxy. By modeling the observed phase-space density of stream stars along with the host galaxy potential it is hoped that we may infer the 3D mass distribution of the host. Many tidal streams are observed around the Milky Way, M31, and other nearby galaxies; the known streams span a large range of distances --- from $\approx10$ to $100~{\rm kpc}$ --- and progenitor masses --- from $\approx$10$^3$ to $\approx$10$^8~\msun$ in stellar mass --- \citep[][]{cite many}. There has been extensive work on developing methods to use data from these streams to measure properties of Milky Way's dark matter halo. These methods span a range of complexity from orbit-fitting, to \emph{Streakline} or particle-spray models, to action-space density modeling, to N-body simulations \citep[see, e.g., the introduction of][]{kuepper15}. All methods have been tested in some way on simulated observations of data and these tests typically demonstrate the recovery of parameters for analytic, static potential forms. 
% internal question: This is all just about the methods -- do I need sentences or a paragraph on results?

One example of stream modeling in a multi-component (static, analytic) potential was done by \citet{pearson15}, who aimed to reproduce observations of the stellar stream density from the globular cluster Palomar 5 in a single oblate and single triaxial potential using \emph{Streakline} \citep{kuepper12} and N-body models. They used the observed SDSS number density of stream stars, a limited number of radial velocities for stream members, and the sky position, distance, radial velocity, and proper motion of the cluster itself to fit model streams to the data. In the oblate potential (a three-component bulge+disk+spherical halo potential), a thin model stream was easily found that reproduces the observed stellar density morphology of the stream. In the triaxial potential (the potential from \cite{law10}: a three component bulge+disk+triaxial halo fit to Sagittarius stream data), the model streams generically formed large, two-dimensional `fans' of debris near the ends, and no physically reasonable progenitor orbits could be found that reproduced the observed thinness and curvature of the stream given the observational constraints of the present-day position and velocity of the cluster. The result in \citet{pearson15} demonstrates that the morphology of a stream alone can rule out a potential. With an understanding of the circumstances that lead to the differences in stream morphology, this could be a powerful tool for rejecting potentials from positional information alone.

The obvious difference between the two potentials considered by \citet{pearson15} is the extra symmetry of the axisymmetric potential. It is well known that the number of degrees of freedom of a potential plays a critical role in determining the orbit structure of the potential --- Hamiltonians with more than two degrees of freedom generically contain significant chaotic regions. \citet{pearson15} tested the stochasticity of the orbit of the progenitor that produced `fanned' debris by computing the Lyapunov exponent along this orbit but found that it is consistent with being regular over dynamically relevant timescales (many Hubble times). It has been shown previously that along some strongly chaotic orbits, tidal streams do form large, diffuse `fans' of debris \citep[e.g.,][]{fardal14}, however it is unknown how the resultant properties of the debris (e.g., density or length of the stream) depend on the degree of stochasticity. The result from \citet{pearson15} suggests that even weak chaos (as measured by the Lyapunov exponent) may affect the density evolution and therefore observability of tidal streams. Understanding why this occurs and developing a measure to quantify the importance of this enhanced density evolution is a promising new direction to be explored further in this work.

In this work, we study the effect of chaotic diffusion of the fundamental frequencies of individual orbits on the density evolution of tidal debris. We choose a simple, cosmologically motivated model for a triaxial potential, analyze the degree of chaos as computed from single-orbit diagnostics for grids of constant-energy orbits, and compare these results to measures of the density evolution of finite-volume ensembles of orbits (meant to mimic tidal debris). We find that even when the chaotic timescale is predicted to be long for a given orbit, chaos may manifest over much shorter times in small orbit ensembles due to the chaotic diffusion of the individual orbits. For a chaotic orbit, the frequency spectrum of the orbit evolves with time --- for a small ensemble, the spread in frequencies is therefore time-dependent, which could enhance phase-mixing. This idea supports a reevaluation of the importance of chaos in galactic halos and implies that the amount of chaos in a given potential may have significant consequences for the observability and survivability of thin, cold tidal streams. We propose a new measure for determining whether this small-scale frequency diffusion will be important for the morphological evolution of tidal debris with a given spread in frequencies or orbital properties over finite and dynamically relevant times.

This paper is organized as follows. We review relevant nonlinear dynamics in Section~\ref{sec:nldreview}. In Section~\ref{sec:methods}, we describe our choice of potential, method for numerical orbit integration, and introduce the chaos indicators used in this work. Our results are split into four subsections: in Section~\ref{sec:results1} we present iso-energy grids of orbits and discuss the orbit classes and chaotic timescales present in our potential; in Section~\ref{sec:results2} we study the density evolution of small ensembles of orbits around each orbit of the previous section; in Section~\ref{sec:results3} we describe the behavior of chaotic diffusion and use this to explain how chaos is relevant for tidal streams over short times; and in Section~\ref{sec:results4} we propose a new indicator for evaluating whether chaotic diffusion of orbits will be important for the morphological evolution of tidal debris. We discuss the implications of our results in Section~\ref{sec:discussion}, and conclude in Section~\ref{sec:conclusions}.

\section{Review of nonlinear dynamics}\label{sec:nldreview}

To explore the question of if and how chaos manifests in the density evolution of orbit ensembles over timescales much shorter than that predicted from generic chaos indicators, we must first understand the behavior of individual orbits in complex gravitational potentials and the orbital structures in the potential itself (i.e. the strength of resonances and chaos). An orbit in an $N$ degree of freedom (dof) Hamiltonian, $H$, is a set of $2N$ quasi-periodic time series, 
\begin{equation}
(w_1(t),...,w_{2N}(t)) = (q_1(t),...,q_{N}(t),p_1(t),...,p_{N}(t)) \label{eq:coords}
\end{equation}
where $q_i$ and $p_i$ are conjugate coordinates in the sense that
\begin{align}
	\dot{p}_i &= -\frac{\partial H}{\partial q_i}\\
	\dot{q}_i &= \frac{\partial H}{\partial p_i}
\end{align}
for all $t$. If bounded, the motion in any component, $w_i(t)$, can be described as a Fourier sum,
\begin{equation}
	w_i(t) = \sum_k^\infty a_k \, e^{i\,\omega_k\,t} \label{eq:fourier}
\end{equation}
where the $a_k$ are complex amplitudes.

A \emph{regular orbit} is a set of such time series that can be transformed to a special set of canonical coordinates known as angle-action coordinates. In these coordinates, the position variables are angles, $\boldsymbol{\theta}$, that increase linearly with time with rates set by $N$ constant, fundamental frequencies, $\boldsymbol{\Omega} = (\Omega_1, ..., \Omega_N)$. The frequency of a Fourier component in the sum for any individual component of motion (the $\omega_k$ in Equation~\ref{eq:fourier}) are just linear, integer combinations of the fundamental frequencies, $\boldsymbol{\Omega}$ --- that is, for a regular orbit, any Fourier component frequency may be written
\begin{align}
	\omega_k &= \boldsymbol{n_k} \cdot \boldsymbol{\Omega} \label{eq:fourierfreq}
\end{align} %; \, (\boldsymbol{n}_k \in \mathbb{Z}^3)
where $\boldsymbol{n}_k$ is a vector of $N$ integers. The conjugate momentum coordinates --- the actions,  $\boldsymbol{J}$ --- are constants of motion. Even stronger, the actions are isolating integrals and any pair are in involution such that
\begin{equation}
	[J_i, J_j] = 0
\end{equation}
where $[\cdot,\cdot]$ is the Poisson bracket. This implies that for an $N$ dof system, a regular orbit has $N$ independent constants of motion and the motion is therefore restricted to an $N$-dimensional manifold embedded in the 2$N$ dimensional phase space. The topology of angle-action space is toroidal and any regular orbit in an $N$ dof Hamiltonian can be understood as motion on the surface of an $N$-torus. Each set of actions, $(J_1,...,J_N)$ (or frequencies), uniquely labels a torus, and regular orbits are sometimes referred to in terms of their orbital tori. 

\subsection{Orbits in integrable potentials}

A Hamiltonian or potential is said to be \emph{globally integrable} when the number of isolating integrals of motion is equal to the number of degrees of freedom and a transformation to angle-action coordinates may be done globally --- for example, the transformation to angle-action coordinates may be written as a function of arbitrary phase-space coordinates and the functional form is independent of phase-space position \citep[e.g.,][]{goldstein80}. The condition for global integrability is very restrictive and it seems that the likelihood that a Hamiltonian is globally integrable decreases as the number of degrees of freedom increase \citep[e.g.,][]{lichtenberg83}. In a globally integrable potential, the Hamiltonian may be written solely in terms of the actions, $H = H(\boldsymbol{J})$. Galactic potentials are almost certainly not globally integrable but it is useful to understand the orbit structure in integrable systems before extending to more general potentials. Only one globally integrable triaxial mass model is known that generates motion that is qualitatively similar to that seen in Galaxies: the St\"ackel or `Perfect Ellipsoid' model \citep[e.g.,][]{kuzmin73, deZeeuw85}. The four general classes of orbits in this potential are box, inner long-axis tube, outer long-axis tube, and short-axis tube orbits. The tube orbits preserve a sense of rotation about either the long or short axis and are therefore centrophobic, whereas the box orbits may approach arbitrarily close to the center of the potential (centrophilic) and may change sense of rotation about any axis.

The frequencies of a generic orbit are typically incommensurable --- that is, $\bs{n} \cdot \bs{\Omega} \neq 0$ for any integer vector, $\bs{n}$, with reasonable magnitude.\footnote{A more precise condition is stated in terms of a diophantine condition, e.g., $|\bs{n} \cdot \boldsymbol{\Omega}| > \alpha \, |n|^{-\gamma}$ where $\alpha, \gamma>0$.} These conditionally-periodic orbits uniformly cover the surface of an orbital torus. If instead there exists a relation of the form $\boldsymbol{n} \cdot \boldsymbol{\Omega} = 0$ the orbit is referred to as a resonant orbit. Resonant orbits are confined to a surface with lower dimensionality than the surface of an orbital torus, depending on the number of resonance relations obeyed: we refer to orbits that obey a single resonance relation as \emph{uni-resonant} orbits, and if an additional resonance relation exists, \emph{bi-resonant}. Uni-resonant orbits in a triaxial potential are confined to a 2D surface, and bi-resonant orbits are closed 1D curves. The resonant structure of a potential --- the relative importance of particular resonance numbers --- is difficult to compute, but determines the global behavior of orbits in the potential. For a triaxial potential the resonances define planes in frequency-space, but resonance surfaces in action-space are generic surfaces. In plots of frequency ratios, the resonances appear as lines; Figure~\ref{fig:cartoons} (left panel) shows a cartoon portrait of a portion of frequency-space for an integrable potential with example resonance lines, non-resonant, and resonant orbits marked. For an integrable potential, all orbits are regular.

\subsection{Orbits in near- and non-integrable potentials}

The orbit structure of near-integrable potentials can be understood by considering a Hamiltonian that is a small perturbation away from being globally integrable --- that is, a Hamiltonian that may be written
\begin{equation}
	H(\boldsymbol{J}, \boldsymbol{\theta}) = H_0(\boldsymbol{J}) + \epsilon \, H_1(\boldsymbol{J}, \boldsymbol{\theta})
\end{equation}
where $H_0(\bs{J})$ represents an integrable Hamiltonian, and $\epsilon$ is a small parameter that determines the perturbation strength \citep[a description of perturbation theory applied to nonlinear Hamiltonians is given in][]{lichtenberg83}. When $0 < |\epsilon| \ll 1$, resonant surfaces become `thick' resonant layers, within which orbits are qualitatively similar to the parent resonant orbit \citep[e.g.,][]{merritt99}. These resonant layers are then generically surrounded by stochastic layers where chaotic motion occurs (in the vicinity of separatrices). Chaotic orbits are characterized by exhibiting random behavior despite being generated by entirely deterministic equations of motion. Actions strictly do not exist for chaotic orbits and therefore chaos `destroys' orbital tori in certain regions of action-space --- these frequency spectrum of orbits in these regions evolve chaotically with time. 

Figure~\ref{fig:cartoons} (middle panel) shows a cartoon of frequency space for a near-integrable potential (a small perturbation away from the potential in the left panel). Much of the structure that was present in the integrable potential remains in the near-integrable case, but the differences are highlighted. Orbits in the resonant layers surrounding the resonances (grey) are near-resonant orbits that librate around the resonance. Chaotic orbits in the stochastic layer (red) behave erratically depending on the surrounding resonance structure. If the stochastic layer is small and the chaotic orbit is therefore confined, the orbit may behave nearly regular for long periods of time.

For small values of $\epsilon$, many regular orbits survive and only small chaotic regions are introduced, especially where resonances overlap \citep[see][]{chirikov60}. As the strength of the perturbation increases, eventually all tori associated with conditionally-periodic motion will be destroyed. Figure~\ref{fig:cartoons} (right panel) qualitatively shows this phenomenon --- the resonant and resonant-layer orbits may still be regular, but many or all of the conditionally-periodic orbits can be chaotic. As the perturbation strength increases, next the uni-resonant tori are destroyed, and finally the bi-resonant tori --- these are least susceptible to destruction from perturbations \cite[for a more quantitative illustration of this transition from integrability to global chaos, see Figure~9 in][]{valluri98}.

When $\epsilon$ is large, there is no general prediction for the resulting behavior, however it seems that more complicated and physically motivated triaxial potential models for galaxies follow the intuition gained from the small-perturbation picture described above, at least for certain parameter choices \citep[e.g.,][]{valluri98, merritt99}. We therefore expect a large number of regular orbits will survive --- the so-called KAM tori --- however the tori that survive will be separated by regions of chaotic motion. Any transformations to angle-action coordinates must be defined local to each resonance region due to the destruction of tori and chaotic motion which lead to discontinuous changes in orbital properties.

\subsection{The behavior of chaotic orbits in non-integrable potentials}\label{sec:behavior-chaotic}

Chaotic orbits have no orbital actions because they are only strictly bound to their energy hypersurface (if the potential is time-independent). The orbits therefore do not have a single set of fundamental frequencies, but rather the frequencies that describe the character of motion evolve with time. In near-integrable potentials, chaotic diffusion of an orbit in actions and frequencies occurs both across resonance layers (a sort of stochastic libration) and along the stochastic layers that surround the resonances (Arnold diffusion). For weakly chaotic orbits, motion across a resonance layer can occur with a frequency close to the libration frequency of the nearby stable orbits in the resonance layer. Thus, if the resonance libration frequency is comparable (within a factor of a few) to the orbital frequency, motion across a resonance can modulate the frequency spectrum of an orbit over an orbital time. However, the stochastic layers are often bounded in the direction orthogonal to the resonance by other stable, resonant regions so that the frequencies or actions can not change by large factors (unless there is resonance overlap and the motion is strongly chaotic). 

The rate of diffusion along stochastic layers via Arnold diffusion depends on the local resonant structure and is hard to predict. This has been done analytically for simple potentials \citep[e.g.,][]{chirikov79}. For systems with $N>2$ dof, the stochastic layers connect and form an intricate network of stochasticity known as the Arnold web; an orbit that ergodically mixes over its energy hypersurface must traverse this web, though the timescales typical for this phenomenon are many thousands of orbital periods. 

Arnold diffusion is not expected to be significant for most orbits over timescales relevant to galaxies (10s of orbits), however chaotic motion across resonances can occur over short times. If a stochastic orbit is semi-bounded by surrounding resonances, the allowed volume in which it can explore (in frequency space) is often quite small so that the orbit may appear to be approximately regular. However, this small-scale evolution may be important for tidal debris which can have small intrinsic spreads in frequencies.

\subsection{Mixing of orbit ensembles}\label{sec:chaotic-mixing}

An ensemble of regular orbits (e.g., tidal debris) will phase-mix due to (small) differences in the fundamental frequencies of the orbits. Generically, a small ensemble of chaotic orbits will lose coherence much faster than for regular orbits \cite[see, e.g.,][]{kandrup94, merritt96, kandrup03}, however this depends on the details of the resonant structure around the ensemble and are thus difficult to predict. For example,  ensembles of orbits `stuck' between resonances may quickly spread to fill the allowed volume via the short-time chaotic diffusion processes, but then the orbits must escape this confinement and diffuse through the Arnold web to reach a fully mixed state \citep{merritt96}.

In this work, we investigate the consequences of this short-time but small-scale frequency evolution and hypothesize that this may explain the enhanced density evolution of tidal debris around weakly chaotic regions where chaotic timescales are predicted to be long \citep[e.g.,][]{pearson15}. We then discuss how this would affect our understanding of the coherence and density evolution of tidal streams.

\section{Numerical methods}\label{sec:methods}

Our goal is to map the orbit structure of arbitrary (galactic) potentials, with an emphasis on identifying the chaotic orbits and understanding the evolution of these ensembles of orbits over short times. In particular, we aim to understand how this chaos-enhanced density evolution can affect tidal stream morphology. In this section, we describe the methods we will use to detect and quantify the strength of chaos for large grids of orbits.

\subsection{Potential choice}\label{sec:potential}

The density distributions within dark matter halos formed in cosmological N-body simulations are generically triaxial \citep[e.g.,][]{jing02, bett07, zemp09, veraciro11}. With the inclusion of baryonic physics and sub-grid prescriptions for energy input due to supernovae and other feedback mechanisms, the inner potential ($\lesssim$$0.1R_{\rm vir}$ for a $\approx$$10^{12}~\msun$ halo mass) typically becomes more spherical, though the magnitude of this reshaping depends on the particular merger history and star formation efficiency within a given halo \citep[e.g.,][though in Milky Way-like galaxies, baryonic disks will add non-sphericity to the total potential]{dubinski94,kazantzidis04, bryan13, butsky15}. It is less clear what happens to the outer halo.

 \citet[][hereafter JS02]{jing02} found that a triaxial generalization of the NFW density profile \citep{navarro96} generates excellent fits to the density distributions within haloes in their high-resolution (dark-matter-only) N-body simulations, and they provide probability distributions for the axis ratios of a large sample of these halos. JS02 find median axis ratios of $c/a \approx 0.55$ and $b/a \approx 0.77$ where $a$ is the major axis, $b$ the intermediate, and $c$ the minor axis.\footnote{Note that JS02 use the opposite notation so that $c$ is the major and $a$ is the minor axis.} These are largely consistent with findings from more recent simulations \citep[e.g.,][]{bett07, veraciro11, butsky15} and consistent with constraints from weak lensing that place a lower limit on minor-to-major axis ratios of $c/a\gtrsim0.5$ \citep{vanuitert12}. JS02 find significant scatter in the distributions of concentration parameter, $c_e$, or scale radius (depending on choice of parametrization). 

All of these parameters are specified in terms of the \emph{density}; for orbit analysis, we need to determine the form of the potential in terms of these parameters, which, in general, requires numerical integration of the density at each position of interest. For computational efficiency, many authors instead express the triaxiality in the form of the potential, but this can lead to unphysical situations where the density becomes negative. \citet{leesuto03} derive a perturbative expansion of the potential integral for a triaxial NFW density and show that the expansion is accurate even for modest axis ratios (e.g., the median values shown above). 

In this work, we use the triaxial potential expression from \citet{leesuto03}, parametrized in a slightly different manner. In terms of spherical coordinates\footnote{$(r,\phi,\theta)$ = (radius, azimuth, colatitude)} with the radius normalized by the scale radius, $u = r/r_s$
\begin{align}
	\Phi(u,\phi,\theta) &\approx \frac{v_c^2}{A}\left[F_1(u) + \frac{1}{2}(e_b^2 + e_c^2)F_2(u) + \frac{1}{2} [(e_b\sin\theta \sin\phi)^2 + (e_c\cos\theta)^2] F_3(u) \right]\label{eq:potential}\\
	A &= \left(\ln2 - \frac{1}{2}\right) + \left(\ln2-\frac{3}{4}\right) (e_b^2 + e_c^2)
\end{align}
where $e_b = \sqrt{1 - (b/a)^2}$, $e_c = \sqrt{1 - (c/a)^2}$, and $v_c$ is the circular velocity at the scale radius, $r_s$, for the spherical case. The functions $F_i(u)$ are given in the appendix of \cite{leesuto03}. We chose $r_s=20~{\rm kpc}$ and $v_c = 175~{\rm km}~{\rm s}^{-1}$ by taking the mean halo concentration for a ${\rm M}_{vir} \approx 10^{12}~\msun$ halo, $c_e\approx5$, from \cite{jing02} and by assuming $R_{vir}\approx200~{\rm kpc}$. Figure~\ref{fig:potential} shows equipotential contours of this potential in projection, and Table~\ref{tbl:potential} summarizes the potential parameters.

\begin{table*}[ht]
\begin{center}
	\begin{tabular}{ c  c }
	         Parameter & Value\\\toprule
		$v_c$ & 175~km~s$^{-1}$\\
		$r_s$ & 20~kpc\\
		$a$ & 1\\
		$b$ & 0.77\\
		$c$ & 0.55\\
		\bottomrule
		\end{tabular}
	\caption{Summary of parameters for the triaxial NFW potential (Equation~\ref{eq:potential}) used in this work. Triaxiality is introduced in the density (rather than the potential) to ensure that the density is physical at all radii. Velocity scale, scale radius, and axis ratios are chosen to match the median halo parameters for a $M_{\rm vir} \approx 10^{12}~\msun$ halo from \citep{jing02}. \label{tbl:potential}}
\end{center}
\end{table*}

This potential is a simple (and unrealistic) model for the total potential of a Milky-Way-like galaxy, however it represents a conservative choice for exploring the structure of orbits in the halos of such galaxies. Realistic galactic potentials will have a significant component due to the disk and bulge, may have twisting inertia tensors \citep{romanowsky98}, radially changing axis ratios \citep[e.g.,][]{kazantzidis04,debattista08,veraciro11,butsky15}, significant substructure \citep{moore98,zemp09}, or time dependence \citep[either from bulk rotation, mass growth, mergers, etc.; see, e.g.,][]{bailin05}. We expect inclusion of any of these effects to increase the amount and influence of chaos; see the Discussion (Section~\ref{sec:discussion}) for a few simple demonstrations. 

\subsection{Orbit integration}\label{sec:integration}

We use the Dormand-Prince 8th-order Runge-Kutta scheme \citep{prince81} to integrate orbits in the above potential. Specifically, we use a \texttt{Python} wrapper over the \texttt{C} implementation by \cite{hairer93}. For all orbits we ensure that energy is conserved to $|\Delta E/E_0| \leq 10^{-8}$ by the end of integration, however most orbits conserve energy to a part in $\approx$$10^{-13}$. Unless otherwise specified the integration timesteps are chosen so that there are 512 steps per strongest orbital period component, but the integrator uses adaptive stepping between each main step in order to satisfy a specified tolerance (we set the absolute tolerance to $\approx$100 times machine precision, $\texttt{atol} = 10^{-13}$). 

\subsection{Lyapunov exponents} \label{sec:lyap}

The most well-known method for assessing chaotic motion is to analyze the Lyapunov spectrum or maximum Lyapunov exponent (MLE) of an orbit. The MLE measures the mean rate of divergence of two infinitesimally separated orbits and is only strictly defined in terms of a limit that goes to infinite time. Thus, we can never truly compute the MLE and it can take integration for many thousands of orbital periods to compute a converged numerical approximation of the MLE for a moderately chaotic orbit. In this work, we use the algorithm introduced by \cite{wolf85} for computing the MLE (for more a more detailed description of this algorithm, see Appendix~\ref{sec:lyapapdx}).

The MLE, $\lambda_{\rm max}$, is interpreted as a rate that quantifies the exponential divergence of infinitesimally close chaotic orbits. It is therefore useful to consider the corresponding $e$-folding time by inverting the rate. We normalize the Lyapunov time by the maximum orbital period\footnote{The maximum of the three orbital periods corresponding to the three fundamental frequencies (see Section~\ref{sec:naff}).} of a given orbit, $T_{\rm max}$, so that the time is in units of number of orbital periods
\begin{equation}
	t_{\rm \lambda} = \frac{1}{\lambda_{\rm max} \, T_{\rm max}}
\end{equation}
We will use this as the prediction from the Lyapunov exponent for the timescale over which chaos should be dynamically important for a given orbit. 

\subsection{Frequency diffusion rate}\label{sec:naff}

Bounded, regular orbits in a triaxial potential have three fundamental frequencies, $\bs{\Omega}$, that determine the periodic behavior of motion. The motion in any canonical coordinate can therefore be decomposed as a Fourier sum (Equation~\ref{eq:fourier}) where the Fourier frequencies are linear, integer combinations of the fundamental frequencies (Equation~\ref{eq:fourierfreq}). \cite{laskar93} introduced a method for recovering the fundamental frequencies of an orbit that effectively uses a filtered fast-Fourier transform (FFT) of complex combinations of the motion (e.g., $x(t) + i v_x(t)$). It has been shown that the accuracy in determining the frequencies converges much faster using a filtered FFT compared to the typical $\inttime^{-1}$ expected for a standard FFT \citep{laskar99}, where $\inttime$ is the integration time. The FFT is iteratively searched for the strongest frequency component, which is subtracted from the spectrum after each iteration. This procedure generates a table of frequencies for each component of motion which must then be searched for the three fundamental frequencies.

This method is referred to as `Numerical Approximation of Fundamental Frequencies' (NAFF) and has been used extensively in planetary dynamics \citep[e.g.,][]{laskar93b, laskar96} and galaxy dynamics \citep{papaphilippou98, valluri98}, especially in the study of orbits in triaxial systems. We have implemented and tested a version of this procedure in the \project{Python} programming language. Our implementation differs slightly from the original definition and from that used in \cite{valluri98}: we have found that using a higher order window function \citep[e.g.,][]{hunter02} of the form
\begin{equation}
	W(\tau=t/\inttime) = \frac{2^p \, (p!)^2}{(2p)!} \left( 1 + \cos \pi \tau\right)^p
\end{equation}
with $p=2$ allows for more reliable determination of frequencies from strongly chaotic orbits. We refer to this slight modification of the algorithm as \superfreq\ \citep{superfreq} and the code is open-source and publicly available on \project{GitHub} at \url{https://github.com/adrn/SuperFreq}. For more details about the algorithm, see Appendix~\ref{sec:naffapdx} or \cite{laskar88, laskar93, papaphilippou96}.

If an orbit is chaotic, the motion can no longer be expanded in terms of a single set of fundamental frequencies. For a weakly chaotic orbit, the orbit may appear consistently periodic over long windows of time. \superfreq\ will pick out a set of frequencies for chaotic orbits that correspond to the largest peaks in the power spectrum of the orbits, however these peaks will change location and amplitude with time. For more strongly chaotic orbits, the power spectrum will be quite noisy and the peak frequencies may change erratically when comparing two consecutive sections of orbit. The frequencies picked out by \superfreq\ for such orbits will therefore represent the average periodic nature of the orbit over a given integration window. We define the fractional frequency diffusion rate per orbital period, $\mathcal{R}$, in the $k$th fundamental frequency as
\begin{equation}
	\mathcal{R}_k = \frac{\Omega_{k}^{(2)} - \Omega_{k}^{(1)}}{\Omega_{k}^{(1)} \, N} \label{eq:fdrate}
\end{equation}
where the upper index refers to the two consecutive sections of orbit and $N$ is the number of orbital periods in a single section of orbit. By inverting this rate, we can compute the timescale over which we expect order-unity changes to the fundamental frequencies: the \emph{frequency diffusion time} is defined as
\begin{equation}
	t_\Omega = \, (\max_{a_k} \, \mathcal{R}_k)^{-1} \label{eq:fdtime}
\end{equation}
where the maximum is taken with respect to the corresponding amplitudes, $a_k$, of the fundamental frequency components \citep[see][]{valluri12}, and the time is in units of number of orbital periods, ${\rm T}$. 

For a small ensemble of orbits (e.g., tidal debris), the relevant timescale is the time over which the change in frequencies for a single orbit is comparable to the spread of frequencies in the ensemble. We can estimate this timescale by multiplying the frequency diffusion time by a factor equal to the fractional spread in frequencies of the debris. For example, a globular cluster typically has $\approx$1\% spreads in fundamental frequencies, so by multiplying the frequency diffusion time by $f = 0.01$, we can estimate the time (in number of orbital periods) over which we expect the frequencies to evolve by this amount.

% =====================================================================
%	Results
%
\section{Results}

In Section~\ref{sec:results1}, we generate grids of orbits in the potential described above to map the orbit structure of the potential. We classify each orbit in terms of the strength of chaos along the orbit as computed using the Lyapnov and frequency diffusion times of Section~\ref{sec:methods}. With this initial classification, in Section~\ref{sec:results2} we follow the evolution of ensembles of orbits generated around each orbit in the initial grid. We find that the configuration-space density of ensembles around weakly chaotic orbits evolve faster (e.g., in mean density) than expected given the timescales over which chaos is computed to be relevant for the parent orbits. In Section~\ref{sec:results3}, we explain this phenomenon in the context of how chaotic diffusion occurs (e.g., Section~\ref{sec:behavior-chaotic}).

\subsection{Part I: Lyapunov and frequency diffusion times}\label{sec:results1}

We generate an isoenergy grid of initial conditions along the $xz$ ($y=0$) plane\footnote{$x$ is the major and $z$ the minor axis.} with energy (per unit mass), $E$, chosen to span a range of distances comparable to the scale radius of the potential ($E=-(397.2~{\rm km~s}^{-1})^2$ in physical units; see Table~\ref{tbl:potential}). We fix $v_x = v_z = 0$, and compute $v_y$ from the energy. The majority of the orbits generated on this grid are tube orbits, which preserve a sense of rotation about one of the principal axes of the potential and are generally centrophobic. Tube orbits are generally less stochastic than box orbits, which tend to plunge deep into the inner regions of the potential where the density profile is steep, however there are appreciable numbers of stochastic tube orbits. The major classes of stable tube orbits circulate about either the major or minor axis. This grid generates all of the major orbit classes (short-axis tubes, inner long-axis tubes, outer long-axis tubes, stochastic intermediate-axis, and box orbits), but the most numerous orbits in this grid are the short-axis and outer long-axis tubes. Thin tidal streams may preferentially form along tube orbits rather than box orbits because of the faster disruption and debris diffusion expected for stellar systems on radially plunging orbits.

Figure~\ref{fig:icmap} shows the grid of initial conditions in the $xz$ plane --- each pixel in this grid represents an orbit, and the major orbit families are labeled. Points are colored (black / white) whether the orbit acts like a tube orbit or a box orbit over $\approx$50 orbital periods. Figure~\ref{fig:icorbits} show sample orbits from each of the main, labeled orbit families of the initial condition grid.

We integrate all orbits in the grid for 10000 orbital periods and use the method described in Section~\ref{sec:lyap} to compute the Lyapunov exponents. Figure~\ref{fig:lyapmap} again shows the grid of initial conditions, but now the logarithm of the Lyapunov time (in units of number of orbital periods) is mapped to greyscale intensity. The darker pixels have shorter Lyapunov times and are more chaotic. By fixing the integration time (in units of orbital periods, $\periods$), the MLE cannot detect weak chaos and the majority of orbits appear to be regular because they have exceedingly long Lyapunov times (all white points have $t_\lambda \gtrsim 1000~\periods$). 

For each orbit, we also separately integrate for $\approx$80 orbital periods and use \superfreq\ to compute the fundamental frequencies for the two consecutive sections of 40 orbital periods. We have found that this is the shortest integration window for which we are able to successfully recover frequencies using \superfreq\ for $>$99.9\% of the orbits. Figure~\ref{fig:freqdiff} shows the same grid of initial conditions as in Figure~\ref{fig:lyapmap}, but now the greyscale intensity is set by the logarithm of the (order-unity) frequency diffusion time. The darker pixels have shorter frequency diffusion times and are more chaotic. This map reveals the intersection of this particular energy hypersurface with the rich structure of resonant surfaces present in this potential and highlights the accuracy of \superfreq\ --- weak chaos (grey) is detectable over just 10's of orbital periods, compared to the many thousands of orbits it would take to detect such features with the maximum Lyapunov exponent. The tube orbits in this potential are mostly regular or only mildly chaotic --- the largest regular regions are associated with the short-axis and long-axis tube orbits --- however islands of stronger chaos do appear, especially at the intersections of resonances where resonance overlap occurs. 

The strongest chaotic regions (black) appear in both of the above grids (Figure~\ref{fig:lyapmap} and Figure~\ref{fig:freqdiff}). Some of the weakly chaotic unstable resonances do appear in the Lyapunov time map, especially where resonances overlap --- for example, near $(x,z) \approx(13,7)~{\rm kpc}$ and $(x,z) \approx(31,13)~{\rm kpc}$ where there is a slight hint of weak chaos (grey) in Figure~\ref{fig:lyapmap}. The details of the resonant structure is revealed in the frequency diffusion map from integrations of just 80 orbital periods. 

While there is rich structure and a significant number of weakly chaotic orbits, the majority of the orbits have estimated chaotic timescales corresponding to thousands of orbital periods and are thus not expected to be relevant for tidal stream evolution. In the next section, we analyze the density evolution of finite-volume ensembles of orbits around each orbit in the above grids in order to study the effect of weak chaos on tidal debris. We then compare the density evolution of the ensembles to the single-orbit chaos indicators computed in this section.

\subsection{Part II: Ensemble properties and mixing} \label{sec:results2}

The Lyapunov time and frequency diffusion rate measure the timescales over which chaos is relevant for a given orbit --- that is, these quantities are measures of how infinitesimal deviations will diverge on average from some parent orbit of interest, or of how long it takes for the frequencies of a single orbit to change by some amount. Tidal debris is disrupted from progenitor systems with finite spreads in orbital properties (e.g., energy). For a disrupting, globular-cluster-scale progenitor, the typical energy dispersion of the debris is 0.1--1\% of the progenitor orbital energy \citep[assuming masses of $10^4$--$10^5$~\msun;][]{johnston98}, but for a dwarf-galaxy-scale progenitor, the dispersion can be $\gtrsim$10\%. In this section, we ask whether  the Lyapunov or frequency diffusion time predict the timescale over which a finite phase-space volume (e.g., tidal debris) stays coherent.

It is computationally intractable to run full N-body simulations for the large grid of orbital initial conditions of the previous section and we therefore take a simplified approach for studying how finite-volume debris spreads along each of these orbits. We instead consider small ensembles of particles meant to represent debris disrupted from a single tidal disruption event. For a given set of orbital initial conditions --- the `parent' orbit --- we find the phase-space position of the nearest pericenter by integrating forward for a short time until a minimum is found, initialize a small ensemble of test particle orbits around this position, then follow the orbits of all test particles for a specified integration time. The physical scale of the ensemble is set by the tidal radius in position and the velocity scale in velocity and are therefore set by the mass of the progenitor \citep[e.g.,][]{johnston98, apw14}. If $(\bs{x}_0,\bs{v}_0)$ are the parent orbit initial conditions at pericenter, then $(\delta\bs{x}_i,\delta\bs{v}_i)$ is the deviation vector of the $i$th particle and the magnitude of the offsets are assumed to be Normally distributed away from the parent orbit:
\begin{align}
	\delta\bs{x}_i &\sim \mathcal{N}(0, r_{\rm tide})\\
	\delta\bs{v}_i &\sim \mathcal{N}(0, \sigma_v)\\
	r_{\rm tide} &= \|\bs{x}_0\| \left(\frac{m}{3M(<\|\bs{x}_0\|)}\right)^{1/3} \\
	\sigma_v &= \|\bs{v}_0\|\left(\frac{m}{3M(<\|\bs{x}_0\|)}\right)^{1/3} 
\end{align}
where $M(<r)$ is the mass enclosed of the host potential within radius $r$, $m$ is the mass scale of the `progenitor,' and $\|\cdot \|$ is the Euclidean norm. We take $m=10^4~\msun$ to represent globular-cluster-like progenitors, and use the spherically-averaged enclosed mass of the host potential to estimate the above debris scales. 

We start by considering three particular orbits chosen from the orbit grid of Section~\ref{sec:results1}: a regular orbit (R), a weakly chaotic orbit (W), and a strongly chaotic orbit (S). The orbits were chosen from a single column of the orbit grid based on their frequency diffusion time to sample a range of possible values, and the initial conditions and chaos diagnostics are listed in Table~\ref{tbl:three-orbits}. Figure~\ref{fig:ensembles} shows final positions of test-particle ensembles initialized around the three orbits described above and evolved for 16 orbital periods. A thin `stream' forms on the regular orbit (left column), a more diffuse `stream' on the mildly chaotic orbit (middle column), and a `fanned' stream on the strongly chaotic orbit (right column). Given the long Lyapunov and frequency diffusion times of the mildly chaotic orbit, it is surprising that the density evolution of the ensemble on this orbit appears to be more diffuse than the regular orbit stream. 

We verify that these orbit ensembles capture the nature of more realistic stream formation by running N-body simulations of globular-cluster-mass progenitor systems on these same three orbits. We use the Self-Consistent Field (SCF) basis function expansion code \citep{hernquist92} to run the simulations, which we set up to run from apocenter-to-apocenter (rather than pericenter-to-apocenter as in the ensembles), but finish with the progenitor in the same location as the parent orbit in the ensemble evolution described above. Figure~\ref{fig:nbodysims} shows the final particle distributions rotated so that the angular momentum of the progenitor orbit is aligned with the $z$-axis. From a comparison of Figures~\ref{fig:ensembles} and \ref{fig:nbodysims}, it is clear that the morphology of the ensembles is visually similar to the `oldest' (first stripped) debris in the N-body simulations.

\begin{table*}[ht]
\begin{center}
	\begin{tabular}{c | c c c c c c | c c }
		{\bf ID} & $\bs{x}$ & $\bs{y}$ & $\bs{z}$ & $\bs{v_x}$ & $\bs{v_y}$ & $\bs{v_z}$ & $\bs{t_\lambda}$ & $\bs{t_\Omega}$ \\\toprule
		R & 29.10 & 0 & 5.10 & 0 & 168.07 & 0 & $>10^3$ & $>10^8$\\
		\midrule
		W & 29.10 & 0 & 17.70 & 0 & 125.49 & 0 & $\approx55$ & $\approx8 \times 10^4$\\
		\midrule
		S & 29.10 & 0 & 22.10 & 0 & 99.45 & 0 & $\approx3.5$ & $\approx6 \times 10^3$\\
		\bottomrule
		\end{tabular}
	\caption{Orbits from the $xz$ grid with a range of chaotic timescales --- R is a regular orbit, W a weakly chaotic orbit, and S a strongly chaotic orbit. Positions ($x$, $y$, $z$) are given in kpc, velocities ($v_x$, $v_y$, $v_z$) in km~s$^{-1}$, and times ($t_\lambda$, $t_\Omega$) in number of orbital periods. Recall that the frequency diffusion time, $t_\Omega$, is the time over which we expect order-unity changes in the fundamental frequencies, hence why the timescales appear quite long. \label{tbl:three-orbits}}
\end{center}
\end{table*}

Visual inspection of Figures~\ref{fig:ensembles} and \ref{fig:nbodysims} suggests that chaotic mixing of small orbit ensembles affects the configuration-space evolution of an ensemble over short times, even when the predicted chaotic timescales (from the Lyapunov and frequency diffusion times) are long and therefore that the mean, single-orbit chaos indicators are not well-suited for determining the importance of chaotic diffusion on tidal stream evolution. As a quantitative measure of this enhanced density evolution, Figure~\ref{fig:ensemble-density} shows the evolution of the configuration-space density for ensembles evolved around the three orbits (R, W, S) described above. At each time step, we use kernel density estimation (KDE)\footnote{We use an implementation from the Python package \texttt{scikit-learn} \citep{scikitlearn}.} with the ensemble of particle positions to estimate the configuration-space density field. We use an Epanechnikov kernel with an adaptive bandwidth: at each evaluation of the density, we use 10-fold cross-validation to find the optimal kernel bandwidth. We evaluate the density at the positions of each particle, $\rho_i$, and compare to the mean initial density, $\mean{\rho_0}$. Figure~\ref{fig:ensemble-density} shows histograms of the logarithm of $\rho_i/\mean{\rho_0}$ for each of the three orbit ensembles at three different times: the initial density distribution (labeled $0~\periods$), and after evolution for 4 and 16 orbital periods. The distributions all shift towards lower densities as time progresses, however other subtle effects are noticeable. The regular orbit distribution stays relatively compact at all times indicating that the density stays fairly uniformly high throughout the stream. For the weakly chaotic orbit, the distribution is somewhat wider with a larger tail to lower densities --- this is the debris that begins to fan out. The strongly chaotic orbit evolves much faster to low density. 

The density evolution can be complex, however with a single measure that characterizes the ensemble density evolution we can map the entire grid of orbits used in previous sections (Figure~\ref{fig:icmap}). The density distributions do not change shape dramatically, even for the strongly chaotic ensemble. Figure~\ref{fig:densities} shows the evolution of the mean density of the ensemble particles computed at 256 evenly-spaced intervals from the initial ensemble distribution to the distribution after 16 orbital periods. It is clear that the density of the regular orbit ensemble evolves slower than the weakly chaotic ensemble, and the strongly chaotic ensemble quickly becomes diffuse; the mean density does not capture the full nature of the density evolution is apparently sensitive to the enhanced density evolution observed along the weakly chaotic orbit seen in Figure~\ref{fig:ensemble-density}.

Motivated by the noticeable discrepancy between the regular and weakly chaotic orbit ensembles --- even though the chaotic timescale of the weakly chaotic ensemble parent orbit is thousands of orbital periods --- we compute the final mean density for orbit ensembles generated around each orbit in the grid described in Section~\ref{sec:results1}. For all ensembles, we integrate the orbits for 16 (parent) orbital periods and store the initial and final values of the mean, configuration-space density. Figure~\ref{fig:ensemblemap-meandensity} again shows the grid of initial conditions (Section~\ref{sec:results1}, Figure~\ref{fig:icmap}), but now the greyscale indicates the ratio of the final mean density to the initial density. Much of the structure that is visible in the frequency diffusion time map (Figure~\ref{fig:freqdiff}) is again visible in this map of the density evolution of orbit ensembles. However, it is surprising that these features are present: the chaotic timescales predicted from both the Lyapunov and frequency diffusion times were typically 100s to 1000s of orbital periods for many of these resonance features.

We conclude from these experiments that the degree of chaos is an indicator that ensembles of orbits will mix faster than regular phase-mixing, however the Lyapunov time and frequency diffusion time are not good predictors for the timescale over which this mixing will occur. To understand this discrepancy, we next explore why this occurs (Section~\ref{sec:results3}, in the context of Section~\ref{sec:chaotic-mixing}) and develop a new indicator that can be tested over a finite number of orbital periods to predict whether chaotic diffusion will enhance the mixing of orbit ensembles (Section~\ref{sec:results4}).

\subsection{Part III: Short-time frequency evolution}\label{sec:results3}

Why does chaos manifest itself after just 16 orbital periods around orbits with predicted chaotic timescales equal to hundreds or thousands of orbital periods? Both the Lyapunov exponent and the frequency diffusion rate measure mean, long-term rates of chaotic diffusion. If a weakly chaotic orbit is confined (by other nearby, non-overlapping resonances or stable regions), the mean drift of an orbit in frequency space may be small if computed over timescales long compared to the orbital time but short compared to the Arnold diffusion time. For the potential considered in this paper, it turns out that these are comparable to the timescales for which the frequency diffusion rate of the previous section were computed ($\approx$100-1000 orbital periods). This window length was chosen in order to have a strong enough signal in the frequency spectrum of chaotic orbits to measure the most significant frequencies using \superfreq\, however, it erases the small-scale variation of the frequencies. 

As a demonstration of the small-scale frequency modulation, we consider again the three orbits of Section~\ref{sec:results2} (initial conditions are listed in Table~\ref{tbl:three-orbits}). Figure~\ref{fig:fdiff-zooms} shows zoomed regions of the frequency diffusion time map of Figure~\ref{fig:freqdiff} centered on the three orbits --- the pixel positions are now shown relative to the three parent orbits. The frequency diffusion times shown in these grids were computed with two consecutive windows of length 40 orbital periods. The resonant structure around the mildly chaotic orbit (white lines in panel W) bound the orbit to a small region, whereas for the strongly chaotic orbit (S), the entire region surrounding the orbit is strongly chaotic. To resolve the short-time behavior of the frequency diffusion (corresponding to motion across or around resonance layers) we compute the frequencies in a series of rolling windows along each orbit. We use a window with a width equal to 40 orbital periods and shift the window by an orbital period between each calculation of the most significant frequencies --- again, this is the smallest window size we could choose before \superfreq\ begins to occasionally fail to recover the frequencies for the strongly chaotic orbit.  Figure~\ref{fig:three-orbits-freqs} shows plots of the frequency ratios as computed in each window of time --- plotted are the percent deviations of the frequency ratios from the initial value. The difference between the first window (green +) and the last window (red x) is 50 orbital periods. The frequencies of the weakly chaotic orbit (panel labeled $W$) evolve quickly but are bounded to a small volume with fractional size $\approx$0.3\% (presumably by nearby resonant surfaces). This is comparable to the frequency spread in globular cluster tidal debris (0.1--0.5\%). %\todo{apw}{check numbers in previous 2 sentences}

We see now where the discrepancy between chaotic timescales and the observable effects of chaos arise in tidal debris: even if the large-scale diffusion of frequencies is slow, an orbit may explore a  region of frequency space over much shorter times. The frequency diffusion time (Equation~\ref{eq:fdtime}) is an estimate of the time over which the mean value of the frequencies evolves, however the \emph{variance} of the frequencies over short times is dynamically relevant for small ensembles. This small-scale variability is insignificant for the evolution of global structure in, for example, an elliptical galaxy, but is signifiant for the morphological evolution of tidal debris with small spreads in frequencies. 

\subsection{Part IV: Proposed indicator of chaotic density evolution}\label{sec:results4}

We repeat the above analysis of the short-time evolution of frequencies for all orbits in the orbit grid used above (Section~\ref{sec:results1}): for each orbit, we integrate for 80 orbital periods, but compute the frequencies now in a rolling window along this integration time. We fix the window width to 40 orbital periods and shift the window by 0.5 orbital periods between each computation of the frequencies. For each orbit, we have the three frequencies (and amplitudes), $k=1,2,3$, computed at times $j$. We compute the fractional amplitude of the frequency variation for a given orbit as
\begin{equation}
	A_\Omega = \max_{a_k} \left(\frac{\max_j \Omega^{(j)}_{k} - \min_j \Omega^{(j)}_{k}}{2 \, \mean{\Omega^{(j)}_{k}}_j}\right) \label{eq:frac-freq-amp}
\end{equation}
where, as in Equation~\ref{eq:fdtime}, the outer maximum is taken with respect to the amplitude of the frequency in the power-spectrum, $a_k$, and the inner maximum and minimum are taken over time, $j$. This is a measure of the magnitude of the short-time frequency evolution for a single orbit. For tidal debris with small spreads in orbital properties, we expect nearby orbits to have similar behavior; for example, the central marker in Figure~\ref{fig:fdiff-zooms} shows the size of a globular-cluster-like system in configuration space. We therefore expect a small ensemble of orbits in frequency space to expand over short times around even weakly chaotic parent orbits and the debris will therefore appear dynamically hotter in real-space. 

We illustrate this phenomenon by repeating the above experiment for small ensembles of orbits around each parent orbit. As before, we compute the frequencies of each orbit in a series of overlapping windows to resolve the short-time evolution of the frequencies. Figures~\ref{fig:regular-ensemble-freq-evolution}--\ref{fig:chaotic-ensemble-freq-evolution} show the evolution of the frequencies for all orbits in each ensemble generated around the three orbits, R, W, and S. In all panels in each of these figures, whats plotted are the per cent deviations of each frequency ratio from the parent orbit frequency ratios computed in the first time window. The left panels in each figure show these per cent deviations for all ensemble orbits around each of the three orbits. Middle panels show the full evolution of each ensemble orbit over 16 orbital periods, and final panel shows the final per cent deviations for all ensemble orbits (after 16 orbital periods). Note that the axis scales are quite different between each figure --- the strongly chaotic orbit ensemble evolves quickly due to its proximity to many overlapping resonances. 

Figure~\ref{fig:freqvar_map} shows the grid of initial conditions, now colored by the (fractional) amplitude of the short-time frequency variation, $A_\Omega$ (Equation~\ref{eq:frac-freq-amp}). When the fractional excursions of the frequencies are of order $10^{-3}$--$10^{-2}$ or larger, globular cluster debris will appear more diffuse due to enhanced density evolution from fast, small-scale chaotic diffusion. Comparing this to the timescales of Figure~\ref{fig:lyapmap} and Figure~\ref{fig:freqdiff}, it is clear that the small-scale variations of the frequencies of a progenitor orbit over a given window of time provides a better prediction of the resulting morphology of tidal debris. 

\section{Discussion: limitations and future work}\label{sec:discussion}

% Something about this: The timescales in this paper are not meant to be interpreted physically, as the potential is a very crude approximation to the total potential of a Milky-Way-like galaxy. The key point is that even when the mean chaos indicators predict a long chaotic timescale, streams may display enhanced density evolution over much shorter times. Tidal debris is sensitive to the small-scale, short-time evolution of orbital properties (e.g., the fundamental frequencies) induced by weak chaos.

We have shown that the Lyapunov and frequency diffusion times are indicators of chaos and that the frequency diffusion time resolves the detailed resonant structure of gravitational potentials, but the timescales predicted do not capture the importance of chaos for the density evolution of ensembles of orbits meant to mimic tidal debris. We have shown that small-scale but fast chaotic diffusion of orbits can explain this enhanced mixing, and have proposed a new indicator to characterize the importance of this effect over given evolution times. In the sub-sections below, we discuss a few important limitations that remain for exploration in future work.

\subsection{The progenitor mass scale}

We have only considered low-mass progenitor systems such as globular clusters because the intrinsic spreads in fundamental frequencies are small (0.1--0.5\%). Small changes to the frequencies of the orbits of tidal debris disrupted from these progenitors due to chaotic processes will therefore cause observable changes to the real-space morphology of the debris. For more massive progenitor systems, the typical size and velocity dispersion of the debris will be larger and thus the debris morphology will be less sensitive to small changes in orbital properties. The mass-scale of the debris that will display enhanced density evolution depends on the magnitude of weak chaos, which depends on the orbit structure of a given potential. In potentials with more significant chaos, debris disrupted from more massive progenitor systems may also display `stream-fanning.'

\subsection{Potential choice}

The potential considered in this work is `unrealistic' for the total potential of a Milky-Way-like galaxy in that it is static, smooth, and does not contain baryonic components. Simulations of forming galactic disks in cosmological dark matter halos have shown that baryonic feedback and relaxation can significantly change the inner distribution of dark matter and either make the potential more spherical or oblate \citep{dubinski94,kazantzidis04, bryan13, butsky15}. However, the significance of baryonic relaxation or of time-dependence, triaxiality, and substructure on shaping the matter distribution within the Milky Way is largely unknown. Here we briefly summarize future directions for potential models:
\begin{itemize}
	\item \emph{Baryonic components}: \cite[][D08]{debattista08} and \cite[][V10]{valluri10} studied the orbit evolution induced by growing a baryonic disk in dark matter halos with various shapes and orientations (e.g., prolate, triaxial). In general, the authors find that the growth of a disk slowly deforms the orbits of mass tracers (e.g., dark matter particles) and preferentially populates tube orbits. Consequently, the inner shape of the potential becomes more oblate or spherical due to the prevalence of tube orbits. If the inner halos of galaxies are indeed close to spherical or oblate, the majority of orbits will be regular and chaos will be less important, however this is far from conclusive. We have found from simple experiments in superposing potential components that transition regions between potential shapes can actually enhance significant amounts of chaos. Though superposing potentials is not realistic, it is at least suggests that a more careful exploration of potential configurations is required to understand how complex, radius-dependent potential forms affect the amount and significance of chaos in galactic halos.

\item \emph{Time dependence}: Galaxies are certainly not static systems. To first order, galaxies grow in mass --- for example, the spherically averaged mass profile of dark matter halos evolves fairly predictably in cosmological simulations \citep{wechsler02, buist14} after some initial period of stochastic mass growth. The Milky Way has probably had a fairly calm accretion history over the last 6 Gyr and therefore the mass growth may be similar to that seen in simulations. This steady growth most likely does not alter the global structure or shape of the potential. However, we also know from simulations that figure rotation, baryonic feedback, and the accretion and phase-mixing of subhalos do perturb the global state of simulated halos. \cite{deibel11} showed that by adopting pattern speeds comparable to those found in cosmological simulations, figure rotation generally acts to destabilize orbits (rather than stabilize chaotic orbits in the equivalent static potential) and the resulting orbit structure is that most regular orbits are associated with resonances. In future work we will explore the effect of these time-dependent processes on the chaotic dispersal of tidal streams using live potentials from cosmological N-body simulations.

\item \emph{Substructure}: Cosmological simulations predict that dark matter halos are filled with substructure in the form of dark matter subhalos. If they exist, these subhalos may account for up to $\approx$1-10\% of the mass of the dark matter \citep[e.g.,][]{diemand07} and therefore may contribute significantly to and orbit in the large-scale potential of any galaxy. Gravitational scattering due to subhalo interactions has been studied, however in the halos of galaxies where dynamical times are long, the scattering cross-section for \emph{strong} encounters is very small \citep{??}. Instead, the collective effect of the subhalos may instead act as a noise term in the Hamiltonian of any halo orbit. This subhalo-induced colored noise --- which depends on the mass spectrum and distribution of subhalos --- may also act to simply increase the magnitude of chaos along orbits, and destabilize sufficiently non-resonant orbits \citep[see, e.g.,][]{kandrup00}.

\end{itemize}

\subsection{Stream modeling}

Tidal stream modeling is one of the most promising ways to constrain the 3D mass distribution around the Milky Way at distances of 10s to 100s of kpc. Methods that use tidal streams to infer properties of the Galactic potential typically operate by constructing models of the debris distribution using either the present-day phase-space density \citep[e.g.,][]{kuepper12, kuepper15, amorisco and others}, time-of-disruption phase-space density \citep{apw13, apw14}, or the density in angle-action coordinates \citep{sanders14, bovy14}. All of these methods may fail or produce uninterpretable results if modeling globular-cluster streams on mildly chaotic orbits. For each of these methods, it is important to understand the failures and biases introduced by ignoring chaotic orbit evolution.

\subsection{Ophiuchus stream}

The Ophiuchus stream \citep{bernard14, sesar15} appears to be a thin, short tidal stream (deprojected length $\approx$1.5--2 kpc) near the Galactic bulge with no apparent progenitor. At Galactocentric $R \approx 1$ kpc, $z \approx 5$ kpc, the orbits of the stream stars likely pass through the MW disk, feel the triaxiality of the Galactic bar \citep[e.g.,][]{wegg13, wegg15}, and have short orbital periods (relative to streams in the halo). It is possible that the observed debris is the last remnants of the recently disrupted progenitor system \citep{sesar15}, however if a significant number of stars were disrupted on previous pericentric passages, this older debris may be `fanned' and still near the observed portion of the stream. The fanned debris would be much lower surface brightness and thus much more difficult to detect. The detection of this low-surface-brightness component would open up the possibility that enhanced density evolution due to chaos is a dynamically relevant process for thin streams in real galaxies, though \cite{carlberg15} has shown that it is also possible also to get short, high density segments of streams from streams formed on eccentric orbits.

\section{Summary and Conclusions}\label{sec:conclusions}

%The large-scale shapes of galactic gravitational potentials are thought to be triaxial and therefore contain an appreciable number of chaotic orbits. Chaos is not thought to be dynamically important for shaping the global structure of galaxies because the predicted chaotic timescales such as the Lyapunov or frequency diffusion times for orbits in galactic-like potentials are very long (many 1000s of orbital periods). However, these timescales measure the mean evolution of orbital properties such as the fundamental frequencies and do not capture the full nature of chaotic orbit diffusion.

We have considered here a simple triaxial gravitational potential chosen to mimic the median properties of dark matter halos formed in dark-matter-only simulations (or the large-scale properties of halos formed in simulations with baryonic effects). We have numerically computed the magnitude of chaos for a large grid of iso-energy orbits in this potential using two independent methods that have been used extensively to classify and characterize chaotic orbits: 1) the Lyapunov exponent and 2) the frequency diffusion rate. From each of these indicators, we compute a timescale over which chaos is likely to be important and find that the majority of orbits have chaotic timescales greater than 100s of orbital periods, however with the frequency diffusion rate we are still able to resolve weak chaos. We then study the density evolution of small ensembles of orbits generated around each orbit in the grid used for the previous experiment and find that along some orbits classified as weakly chaotic --- with chaotic timescales of 100s of orbital periods --- the orbit ensembles display enhanced density evolution and reach a lower overall density faster than orbit ensembles around nearby regular orbits (which mix due to phase-mixing alone). We explain this apparent discrepancy by considering the nature of chaotic diffusion: the classical chaos indicators are most sensitive to the slow, Arnold diffusion process that can cause large changes to orbital properties, but small-scale frequency evolution occurs over much shorter times as chaotic orbits stochastically diffuse across the stochastic layers that surround many resonances. We propose a new measure of chaos with tidal stream morphological evolution in mind: the amplitude of small-scale, fractional changes in frequencies over a given number of orbital periods along a stream progenitor orbit. When this amplitude is comparable to or larger than the typical spread in orbital properties of tidal debris from the progenitor, the phase-space density of the debris will evolve faster to a state of lower density relative to nearby regular orbits.

Our main conclusions are summarized as follows:
\begin{enumerate}
	\item the Lyapunov time and frequency diffusion time are powerful indicators of chaos, but do not capture the importance of small-scale chaotic diffusion for the density evolution of small ensembles of orbits (tidal debris);
	\item tidal debris becomes diffuse and thus harder to observe on weakly chaotic orbits when the small-scale chaotic diffusion of the fundamental frequencies has a scale comparable to the internal spread in frequencies of the debris;
	\item the details of the enhanced mixing along weakly chaotic orbits depends on the resonant structure of the potential;
	\item the variance of the fundamental frequencies of an orbit over a given time window is a strong predictor of the importance of small-scale chaotic frequency diffusion on the resulting morphology of tidal debris.
\end{enumerate}

Our results provide a clear explanation of how and why the morphology of tidal streams alone can be used to constrain the potential of the host galaxy. The longest thin streams are most valuable for this effort because they have clearly evolved for a long time, but the debris remains compact. For shorter thin streams it will be hard to decouple the unknown evolution time from enhanced density evolution from chaos. The mere existence of thin tidal streams in the halo of the Milky Way either (1) provides useful information about the potential on these scales by, e.g., implying a large degree of regularity, or (2) indicates that the thin, long streams (e.g., GD-1) are on regular orbits. These are not mutually exclusive --- in fact, if the streams are on regular orbits, this would be a powerful way to check or rule out possible potential models by requiring that the progenitor orbits remain regular. 

\acknowledgements
The authors wish to thank Robyn Sanderson, Dan D'Orazio, David Merritt, and the \emph{Stream Team} for useful comments and discussion. APW is supported by a National Science Foundation Graduate Research Fellowship under Grant No.\ 11-44155. This work was partially supported by the National Science Foundation under Grant No. AST-1312196.  This research made use of \project{Astropy}, a community-developed core \project{Python} package for Astronomy \citep{astropy13}. This work additionally relied on Columbia University's \emph{Hotfoot} and \emph{Yeti} compute clusters, and we acknowledge the Columbia HPC support staff for assistance.

\bibliographystyle{apj}
\bibliography{refs}

\appendix
\section{Lyapunov exponents} \label{sec:lyapapdx}

%Using the definition of $w$ from Equation~\ref{eq:coords}, we can write Hamilton's equations as \footnote{A sum is implied with any repeated indices.}
%\begin{equation}
%	\dot{w}_i = \mathcal{J}_{ik}\,\frac{\partial H}{\partial w_k} \label{eq:ham}
%\end{equation}
%where $\mathcal{J}_{ik}$ is the $2N \times 2N$ canonical Poisson tensor (also called the symplectic matrix) defined by
%\begin{equation}
%	\mathcal{J}_{ik} = \left( \begin{array}{c:c} 0 & \ident \\ \hdashline -\ident & 0 \end{array} \right)
%\end{equation}
%with $N$-dimensional identity matrices $\ident$. We will consider a nearby phase-space position, $w_i'$, separated from $w_i$ by an infinitesimal deviation, $\delta w_i$, such that $w_i' = w_i + \delta w_i$. We can expand to linear order in the deviation about the parent orbit and write the equations of motion for the deviation as 
%\begin{align}
%	\dot{\delta w_i} &= \mathcal{J}_{ik} \, \frac{\partial^2 H}{\partial w_k \partial w_m} \, \delta w_m =  \mathcal{J}_{ik}\, D_{km} \, \delta w_m \\
%	&= A_{im} \, \delta w_m
%\end{align}
%where $D_{km}$ is the Hessian matrix evaluated at the parent orbit. The general solution for this equation is
%\begin{equation}
%	\delta w_i(t) = B_{im}(t) \, \delta w_i(0)
%\end{equation}
%where $\delta w_i(0)$ are the initial conditions for the deviation vector and $B_{im}$ is the solution matrix. 
% For chaotic orbits, the maximum eigenvalue of the solution matrix to Eq.~\ref{eq:deviate} is positive real, leading to exponential divergence of nearby orbits. 

If one is only interested in characterizing the degree of chaos, computing the full Lyapunov spectrum for an orbit is often not necessary. It is usually sufficient to compute an estimate of the maximum Lyapunov exponent by estimating the finite-time maximum Lyapunov exponent (FTMLE). Using the definition of $\boldsymbol{w}$ from Equation~\ref{eq:coords}, consider an orbit that is a small deviation away from the parent orbit, $\boldsymbol{w}' = \boldsymbol{w} + \delta\boldsymbol{w}$. If the parent orbit is chaotic, the norm of the infinitesimal deviation should grow exponentially with time with some characteristic rate, $\lambda$,
\begin{equation}
	\|\delta\boldsymbol{w}(t)\| = e^{\lambda \, t} \, \|\delta\boldsymbol{w}_0\|
\end{equation}
\citep[see, e.g.,][]{lichtenberg83,tabor89}. From this expression, we see that
\begin{equation}
	\lambda(t) = \frac{1}{t}\ln \frac{\|\delta \bs{w}(t)\|}{\|\delta \bs{w}_0\|} \label{eq:mle}
\end{equation}
where the maximum Lyapunov exponent is the limit as $t\rightarrow \infty$,
\begin{equation}
	\lambda_{\rm max} = \lim_{t\rightarrow\infty}\lambda(t). \label{eq:lmax}
\end{equation}
Numerically computing this quantity is not trivial because (1) obviously the limit to infinity is not possible and (2) the norm of the deviation vector $\|\delta \bs{w}(t)\|$ is expected to increase exponentially for chaotic orbits, leading to nonlinear evolution of the deviation and numerical problems. To circumvent these issues, it is sufficient to instead start a nearby orbit with some small initial deviation with norm $\delta_0$, integrate for a sufficiently small amount of time, $\tau$, then renormalize the deviation back to the initial norm \citep{benettin76}. There is no general way to determine $\tau$ except to perform convergence tests.
%The following pseudocode outlines this procedure:\footnote{see Gary for Python implementation?}\\
%\begin{algorithmic}[1]
%\State {\bf define} orbital integration timestep, $h$, and number of steps, $K$
%\State {\bf define} initial norm of deviation vector, $\delta_0$, to be sufficiently small
%\State {\bf define} renormalization integration period, $\tau \ll hK$ 
%\For{each timestep when integrating the main orbit, $\bs{w}(t)$}
%\State step forward the orbit and deviation vector orbit by one timestep, $t_{i-1} \rightarrow t_i$
%\If {a normalization timestep}
%\State measure and store the length of the deviation vector, $\delta_i = \|\delta \bs{w}_i\|$
%\State renormalize the length of the deviation vector, $\delta \bs{w}_i = \delta \bs{w}_i (\delta_0/\delta_i)$
%\EndIf
%\EndFor 
%\end{algorithmic}
The FTMLE after a given number of timesteps, $N$, is then estimated as
\begin{equation}
	\lambda_N = \frac{1}{t_N}\sum_i^N \ln \frac{\|\delta \bs{w}(t_i)\|}{\|\delta \bs{w}_0\|} \label{eq:ftmle}
\end{equation}
where the $t_i$ are the times at which the renormalization occurs and the MLE is estimated after a very long time to approximate the limit of Equation~\ref{eq:lmax}.

For most regular orbits, deviations will grow linearly or as a power-law of time. As we have seen in Section~\ref{sec:nldreview}, if the orbit is regular, there exists a local transformation to action-angle variables where the angle variables increase linearly with time, $\theta_i \propto \Omega_i t$. We can look at small variations around the angle-space orbit,
\begin{align}
	\frac{d (\delta \theta_i)}{dt} &= \frac{\partial \Omega_i}{\partial J_k} \delta J_k
\end{align}
These equations are easily integrated:
\begin{align}
	\delta \theta_i(t) &= \delta \theta_i(0) + \left(\frac{\partial \Omega_i}{\partial J_k} \delta J_k \right) t
\end{align}
It's clear then that the norm of the deviation vector grows linearly with time:
\begin{align}
	\|\delta \bs{w}(t)\| &= \left[\sum_i (\delta \theta_i(t))^2 + \sum_i (\delta J_i)^2\right]^{1/2}\\
	&= \left[\sum_i \left(\delta \theta_i(0) + \frac{\partial \Omega_i}{\partial J_k} \delta J_k t\right)^2 + \sum_i (\delta J_i)^2\right]^{1/2}\\
	&\propto t.
\end{align}
From Equations~\ref{eq:mle} and \ref{eq:lmax} it is evident that any deviation vector that grows as a power law with time, $t^k$, will asymptote to 0 from the limit 
\begin{equation}
	\lambda_{\rm max} \propto \lim_{t\rightarrow \infty} k \frac{\ln t}{t} = 0.
\end{equation}
At long times, the numerically computed MLE for a regular orbit should approach 0 as $t^{-1}$. For chaotic orbits, the divergence is exponential, and the limit should converge to the rate of the exponential: the Lyapunov exponent. In practice, the MLE is often estimated as the mean of $\lambda_N$ after the summation diverges from power-law behavior. For weakly chaotic orbits, reliable computation of the MLE may take integration of thousands of orbital periods.

\section{SuperFreq}\label{sec:naffapdx}

\todo{apw}{This section has not been edited carefully...}
\todo{apw}{What do I mean by `more reliable' for $p=2$?}

Figure~\ref{fig:logfreqs} shows a test of the implementation in which we reproduce the frequency map at a fixed energy of an axisymmetric, logarithmic potential \cite[][pg. 260, Figure~3.45]{binneytremaine}. Plotted are the (Cartesian) frequency ratios recovered for a grid of iso-energy, box orbits integrated in the potential
\begin{equation}
	\Phi(x,y,z) = \frac{1}{2}\ln\left(x^2 + (y/0.9)^2 + (z/0.7)^2 + 0.1\right). \label{eq:logpotential}
\end{equation}
In such a map, stable resonances appear as linear over-densities and unstable resonances appear as linear under-densities. The regularity of the points in this map reflects the input grid of initial conditions. Points that appear to be erratically scattered are chaotic orbits where the frequencies are changing with time.

\superfreq\ recovers the fundamental frequencies for an orbit faster (with a fewer number of terms) when the coordinates used are `close' to the angle variables \cite[PL96;][]{papaphilippou96}. PL96 show that a good choice of coordinates for tube orbits are the Poincar\'e symplectic polar coordinates, a set of canonical coordinates similar to cylindrical coordinates. When computing the frequencies for tube orbits, we first align the circulation about the $z$-axis through rotation, transform to Poincar\'e polar coordinates, then use \superfreq\ to measure the fundamental frequencies. We could equivalently use the Cartesian time series, but the convergence of terms is slower (the amplitudes of successive terms decrease slower for Cartesian coordinates). We have tested that our implementation of \superfreq\ returns the same fundamental frequencies in either case for a set of tube orbits. For box orbits, the motion is close to separable in each Cartesian component and we therefore use Cartesian coordinates for estimating the frequencies for these orbits. 

% Figure ??
%\begin{sidewaysfigure}[!p]
%\begin{center}
%\includegraphics[width=\textwidth]{figures/cartoons.pdf}
%\caption{} 
%\label{fig:cartoons}
%\end{center}
%\end{sidewaysfigure}
\begin{figure*}[!p]
\begin{center}
\includegraphics[width=\textwidth]{figures/cartoons.pdf}
\caption{An illustrative demonstration of the orbit structure for integrable (left), near-integrable (middle), and non-integrable (right) potentials in terms of the ratios of the fundamental frequencies. In an integrable potential, only resonant and non-resonant orbits exist, and all orbits are regular (resonances appear as lines in frequency-ratio-space). If the potential is perturbed mildly, resonance layers form around the resonances that host near-resonant orbits which behave like resonant orbits but have an additional frequency corresponding to libration about the resonance. Where resonances overlap or near separatrices, stochastic layers form and orbits will be chaotic. For more strongly perturbed potentials, many of the non-resonant orbits may become chaotic but resonance layers may grow and still remain regular. } 
\label{fig:cartoons}
\end{center}
\end{figure*}

% Figure ??
\begin{figure*}[!p]
\begin{center}
\includegraphics[width=\textwidth]{figures/potential.pdf}
\caption{Equipotential contours for the triaxial NFW potential considered in this work. There are eight contour levels evenly spaced and linear in the value of the potential. } \label{fig:potential}
\end{center}
\end{figure*}

% Figure ??
\begin{figure*}[p]
\begin{center}
\includegraphics[width=0.9\textwidth, trim={0 1cm 0 0}]{figures/ic-map.png}
\caption{ A grid of isoenergy orbits initialized on the $xz$ plane. Each pixel in this image represents a single orbit, and the shading (white/black) indicates the behavior of the orbit over 50 orbital periods: if the orbit preserves a sense of rotation around some axis --- that is, the angular momentum around some axis does not change sign --- the orbit is considered a tube orbit (white), otherwise the orbit is considered a box or chaotic orbit (black). The regions hosting the major orbit types are indicated by text labels. Figure~\ref{fig:icorbits} shows examples of each of these four orbit types. } \label{fig:icmap} 
\end{center}
\end{figure*}

% Figure ??
\begin{figure*}[p]
\begin{center}
\includegraphics[width=\textwidth]{figures/ic-orbits-T.png}
\caption{ Visualization of the four major regular orbit types in a triaxial potential: outer long-axis tubes (O), inner long-axis tubes (I), short-axis tubes (S), and box or boxlets (B). Shown are projections of these sample orbits onto the $xy$ and $yz$ planes and meant only to illustrate the general behavior of these orbit classes.} \label{fig:icorbits} 
\end{center}
\end{figure*}

% Figure ??
\begin{figure*}[p]
\begin{center}
\includegraphics[width=0.9\textwidth, trim={0 1cm 0 0}]{figures/lyap_map.png}
\caption{ The same grid of initial conditions as Figure~\ref{fig:icmap}, but now each pixel is colored by the  logarithm of the Lyapunov time (in units of orbital periods). Orbits are integrated for a total of 10000 orbital periods. Chaotic orbits with $t_\lambda \gtrsim 1000~\periods$ appear regular because the integration time for each orbit is insufficient to resolve weak chaos. Three orbits are highlighted with square markers --- from top to bottom, these are the strongly chaotic (S; white), weakly chaotic (W; black), and regular (R; black) orbits of Table~\ref{tbl:three-orbits} (see Section~\ref{sec:results2}).} \label{fig:lyapmap} 
\end{center}
\end{figure*}

% Figure ??
\begin{figure*}[p]
\begin{center}
\includegraphics[width=0.9\textwidth, trim={0 1cm 0 0}]{figures/fdiff_map.png}
\caption{The same grid of initial conditions as Figure~\ref{fig:icmap}, but now each pixel is colored by the logarithm of the frequency diffusion time (again in units of orbital periods). The frequencies are computed in two consecutive windows, each of which has length equal to 40 orbital periods. The frequencies are measured precisely so that small changes in the frequencies can be detected over just $\approx$10s of orbits. Three orbits are highlighted with square markers --- from top to bottom, these are the strongly chaotic (S; white), weakly chaotic (W; black), and regular (R; black) orbits of Table~\ref{tbl:three-orbits} (see Section~\ref{sec:results2}).} \label{fig:freqdiff}
\end{center}
\end{figure*}

% Figure ??
\begin{figure*}[p]
\begin{center}
\includegraphics[width=\textwidth]{figures/ensembles.png}
\caption{Final particle positions after integrating unbound, globular-cluster-sized ensembles of orbits generated around each of the three N-body orbits (Figure~\ref{fig:nbodysims}). The ensembles each contain 1000 particles and begin at pericenter rather than apocenter because they are meant to represent debris disrupted during a single (the first) pericentric passage. The particle positions qualitatively match the morphology of the `oldest' debris from the N-body simulations, though there are many more particles in these ensembles than are disrupted during the first pericentric passage of the simulations. } 
\label{fig:ensembles}
\end{center}
\end{figure*}

% Figure ??
\begin{figure*}[p]
\begin{center}
\includegraphics[width=\textwidth]{figures/nbody.png}
\caption{ We use the Self-Consistent Field (SCF) basis function expansion code \citep{hernquist92} to run N-body simulations of globular-cluster-mass progenitor systems on the three orbits of Figure~\ref{fig:ensembles}. The progenitor in each simulation is initialized as a $10^4$ particle Plummer sphere and the background triaxial NFW potential is turned on slowly over 250 Myr to reduce artificial gravitational shocking. We start the progenitor systems at apocenter and evolve for $\approx$16 orbital periods so that each simulation finishes again at subsequent apocenter. The mass of the progenitor is set to XX, and the length-scale of the Plummer sphere is set to 20~pc to get $\approx$10\% mass loss over the integration time. Panels show particle positions from the final snapshots of the simulations --- for visualization the positions have been rotated so that the angular momentum of the progenitor at the final snapshot are aligned with the $z$-axis. These simulations confirm that the test-particle orbit ensembles do (qualitatively) capture the nature of the early-stripped tidal debris. } 
\label{fig:nbodysims}
\end{center}
\end{figure*}

% Figure ??
\clearpage
\begin{figure*}[p]
\begin{center}
\includegraphics[width=\textwidth]{figures/ensemble-density.png}
\caption{ Distributions of the values of the estimated configuration-space density field evaluated at the positions of the ensemble particles after 0, 4, and 16 orbital periods for each of the three orbits of Figures~\ref{fig:ensembles} and \ref{fig:nbodysims}. The density distribution of the regular orbit ensemble stays compact and decreases slowly. The density of the weakly chaotic ensemble initially evolves faster to lower density and has a larger tail to lower densities. The strongly chaotic ensemble evolves fast to a state of low density.} \label{fig:ensemble-density}
\end{center}
\end{figure*}

% Figure ??
\clearpage
\begin{figure*}[p]
\begin{center}
\includegraphics[width=\textwidth]{figures/ensemble-densities.png}
\caption{This is an alternate visualization of the density evolution of orbit ensembles around the three orbits of Figure~\ref{fig:ensemble-density}, showing only the evolution of the mean density for the orbits (normalized by the initial density of the ensemble). } 
\label{fig:densities}
\end{center}
\end{figure*}

% Figure ??
\clearpage
\begin{figure*}[p]
\begin{center}
\includegraphics[width=0.9\textwidth, trim={0 1cm 0 0}]{figures/ensemble-map.png}
\caption{\todo{all}{Ignore this for now! Waiting for results from latest run on cluster...}The same grid of initial conditions as Figure~\ref{fig:lyapmap} and other previous figures, but now we consider the density evolution of small ensembles of orbits around each of the orbits in this grid. We generate globular-cluster-scale ensembles of orbits around each `parent' orbit in the grid (Section~\ref{sec:results2}) and follow the density of the ensemble as a function of time using kernel density estimation (KDE) with an adaptive Epanechnikov kernel. Three orbits are highlighted with square markers --- from top to bottom, these are the strongly chaotic (S; white), weakly chaotic (W; black), and regular (R; black) orbits of Table~\ref{tbl:three-orbits} (see Section~\ref{sec:results2}).} 
\label{fig:ensemblemap-meandensity}
\end{center}
\end{figure*}

% Figure ??
\clearpage
\begin{figure*}[p]
\begin{center}
\includegraphics[width=1.1\textwidth]{figures/fdiff_zooms.png}
\caption{Zooms of the frequency diffusion time map of Figure~\ref{fig:freqdiff} around the three orbits highlighted in that figure and used in Figures~\ref{fig:nbodysims}--\ref{fig:ensemble-density} --- as in Figure~\ref{fig:freqdiff}, each pixel represents an orbit, and the greyscale of the pixel is set by the frequency diffusion time. Around each orbit, we initialize a new grid of $\approx$40,000 orbits to resolve the small-scale resonant structure around each orbit. Each orbit in the sub-grid is integrated for 80 orbital periods and the frequency diffusion time is computed from the two consecutive sections of 40 orbital periods. The marker at the center indicates the scale (e.g., tidal radius) of a $M \approx 10^4~\msun$ globular cluster at the given position in the initial condition grid. The regular orbit (R) is near a resonance (white line), the weakly chaotic orbit (W) is surrounded by an intricate resonant structure and may be in a region of resonance overlap, and the strongly chaotic orbit (S) is embedded in a region of strong stochasticity. \todo{apw}{Move colorbar to bottom or top?}}
\label{fig:fdiff-zooms}
\end{center}
\end{figure*}

% Figure ??
\clearpage
\begin{figure*}[p]
\begin{center}
\includegraphics[width=\textwidth]{figures/freq-evolution.png}
\caption{Evolution of the fundamental frequency ratios, $r_{1,3} = \Omega_1 / \Omega_3$ and $r_{2,3} = \Omega_2 / \Omega_3$, computed over 256 orbital periods for the three orbits, R, W, S. Fundamental frequencies are computed with a window width of $\approx$40 orbital periods, and the window is shifted by one orbital period between each computation (each grey point represents the fundamental frequency ratios computed in a single window). Plotted are the per cent deviations of the frequency ratios from the value in the initial window --- the initial value is shown as a green plus sign and the final value is shown as a red x. For the regular orbit (R), the fractional variation is around $10^{-6}$ and thus all points overlap on this scale. The weakly chaotic orbit (W) displays frequency variations comparable to the spread in frequencies in globular-cluster-like tidal debris. The frequency variations of the strongly chaotic orbit (S) can be many 10s of per cent.} 
\label{fig:three-orbits-freqs}
\end{center}
\end{figure*}

% Figure ??
\clearpage
\begin{figure*}[p]
\begin{center}
\includegraphics[width=\textwidth]{figures/ensemble-freq-evolution-regular.png}
\caption{Evolution of the fundamental frequency ratios, $r_{1,3} = \Omega_1 / \Omega_3$ and $r_{2,3} = \Omega_2 / \Omega_3$, over 16 orbital periods for an ensemble of 1000 orbits generated around the regular orbit, R, with mass scale $m=10^4~\msun$ (Section~\ref{sec:results2}). Fundamental frequencies are computed with a window width of $\approx$40 orbital periods, and the window is shifted by one orbital period between each computation (each grey point represents the fundamental frequency ratios computed in a single window). Plotted are the per cent deviations of the frequency ratios from the value in the initial window for the parent orbit. Left panel shows frequencies of each ensemble orbit for first time window. Middle panel shows the evolution of the frequencies over 16 orbital periods. Right panel shows final frequency values. For this orbit (R), all panels are identical because the orbit is regular (the frequencies don't evolve).} 
\label{fig:regular-ensemble-freq-evolution}
\end{center}
\end{figure*}

% Figure ??
\clearpage
\begin{figure*}[p]
\begin{center}
\includegraphics[width=\textwidth]{figures/ensemble-freq-evolution-mildly_chaotic.png}
\caption{Same as Figure~\ref{fig:regular-ensemble-freq-evolution}, but for the weakly chaotic orbit (W). Frequencies now display short-time variations (compare left to right panels). Points in middle panel show the trajectories of each ensemble orbit in frequency-space.} 
\label{fig:mildly_chaotic-ensemble-freq-evolution}
\end{center}
\end{figure*}

% Figure ??
\clearpage
\begin{figure*}[p]
\begin{center}
\includegraphics[width=\textwidth]{figures/ensemble-freq-evolution-chaotic.png}
\caption{Same as Figures~\ref{fig:regular-ensemble-freq-evolution}--\ref{fig:mildly_chaotic-ensemble-freq-evolution}, but for the strongly chaotic orbit (S). Frequencies are now explore a huge volume representing $>$100\% deviations from the initial value of the parent orbit. Frequencies evolve quickly around resonances --- especially the $\bs{n}=(2,-3,0)$, $\bs{n}=(1,-1,0)$, and $\bs{n}=(1,0,1)$ resonances --- which appear as linear features in these panels.} 
\label{fig:chaotic-ensemble-freq-evolution}
\end{center}
\end{figure*}

% Figure ??
\clearpage
\begin{figure*}[p]
\begin{center}
\includegraphics[width=0.9\textwidth, trim={0 1cm 0 0}]{figures/freqvar-map.png}
\caption{The same grid of initial conditions as in previous figures, but now each pixel is colored by the amplitude of short-time frequency variations, $A_\Omega$ (Equation~\ref{eq:frac-freq-amp}). The amplitude is a fractional measurement, so that a value of $10^{-2}$ corresponds to a 1\% change in frequencies. To compute the short-time frequency evolution, the frequencies are computed in a sliding window of width 40 orbital periods, and the window is shifted by half of an orbital period between each computation of the frequencies. The frequency evolution is computed for the equivalent of 16 orbital periods. The weakly chaotic orbit has $A_\Omega \approx 0.2\%$ over just 16 orbital periods --- this explains the enhanced density evolution observed in the N-body simulations and ensemble integrations (Figures~\ref{fig:nbodysims}-\ref{fig:ensembles}). The three orbits, R, W, S, are highlighted with square markers (S is top, W is middle, R is bottom). } 
\label{fig:freqvar_map}
\end{center}
\end{figure*}

\end{document}
